\documentclass{ctexart} % 使用ctexart类,适合中文文档
\usepackage{xcolor} 
\usepackage{enumitem} % 加载enumitem包
\usepackage{verbatim}
\usepackage{listings}
\usepackage{xeCJK}
\usepackage{fontspec}
\usepackage{graphicx}
\usepackage{hyperref}
\usepackage{cleveref}
\usepackage{nameref}


% 定义listings设置
\lstset{
    basicstyle=\ttfamily, % 使用等宽字体
    commentstyle=\color{gray}, % 注释颜色
    keywordstyle=\color{blue}, % 关键字颜色
    frame=single,
    stringstyle=\color{red}, % 字符串颜色
    breaklines=true, % 自动换行
    captionpos=b, % 标题位置在底部
    showstringspaces=false, % 不显示字符串中的空格标记
}

\title{project3}
\author{小组二}
\date{\today}

\begin{document}
\maketitle % 创建标题

\section*{题目要求}

There are a lot of tips telling us that some fruits must not be eaten with some other fruits, or we might get ourselves in serious trouble. For example, bananas cannot be eaten with cantaloupe (哈密瓜), otherwise it will lead to kidney deficiency (肾虚).

Now you are given a long list of such tips, and a big basket of fruits. You are supposed to pick up those fruits so that it is safe to eat any of them.

\subsection*{Input Specification}
Each input file contains one test case. For each case, the first line gives two positive integers: \(N\), the number of tips, and \(M\), the number of fruits in the basket. Both numbers are no more than 100.

Then two blocks follow:
\begin{enumerate}[label=\arabic*.]
    \item The first block contains \(N\) pairs of fruits which must not be eaten together, each pair occupies a line and there is no duplicated tips.
    \item The second block contains \(M\) fruits together with their prices, again each pair in a line. To make it simple, each fruit is represented by a 3-digit ID number. A price is a positive integer which is no more than 1000. All the numbers in a line are separated by spaces.
\end{enumerate}

\subsection*{Output Specification}
For each case:
\begin{itemize}
    \item First print in a line the maximum number of safe fruits.
    \item Then in the next line list all the safe fruits, in increasing order of their IDs. The IDs must be separated by exactly one space, and there must be no extra space at the end of the line.
    \item Finally, in the third line print the total price of the above fruits. Since there may be many different solutions, you are supposed to output the one with a maximum number of safe fruits. In case there is a tie, output the one with the lowest total price. It is guaranteed that such a solution is unique.
\end{itemize}

\section*{算法描述}
这段代码实现了一个用于解决``安全水果选择''问题的深度优先搜索(DFS)算法。问题的核心是从一个大篮子中选择出最多数量的可以安全共同食用的水果,如果有多种选择方法,则选择总价最低的组合。

\subsection*{算法概述}
\begin{itemize}
    \item \textbf{输入处理:}首先读取不可以一起食用的水果对,这些对储存在 \texttt{deficiency} 数组中。如果 \texttt{deficiency[i][j]} 为 \texttt{false},则表示水果 $i$ 和水果 $j$ 不能一起食用。接着读取每个水果及其价格,存储在 \texttt{price} 数组中。
    \item \textbf{初始化:}使用 \texttt{memset} 将 \texttt{deficiency} 初始化为 \texttt{true},然后根据输入调整为 \texttt{false} 表示不能共同食用的水果对。
\end{itemize}

\subsection*{深度优先搜索(DFS)}
\begin{itemize}
    \item 函数 \texttt{DFS} 被用于尝试各种不同的水果组合,从而找到最大数量且成本最低的安全食用组合。
    \item 参数 \texttt{pos} 表示当前选择的水果ID,\texttt{sum} 是到目前为止已选择的水果数量,\texttt{money} 是当前已选水果的总价格。
\end{itemize}

\subsection*{剪枝优化}
\begin{itemize}
    \item 如果从当前水果 \texttt{pos} 开始,即使加上后续可能的最大选择数量,其总和仍然小于已知的最大数量 \texttt{ansNum},则直接返回,不再继续搜索。
    \item 如果当前选择的水果与已选择的任何一个水果有冲突(即不能一起食用),则跳过当前选择。
\end{itemize}

\subsection*{更新最优解}
\begin{itemize}
    \item 如果当前的水果组合数量超过已知的最大数量 \texttt{ansNum},更新 \texttt{ansNum},记录当前组合到 \texttt{ansFruit},并更新总价 \texttt{ansMoney}。
    \item 如果当前的水果组合数量等于 \texttt{ansNum} 但价格更低,则同样更新 \texttt{ansFruit} 和 \texttt{ansMoney}。
\end{itemize}

\subsection*{主函数中的循环}
从每个可能的水果开始,使用 \texttt{DFS} 尝试构建有效的水果组合,从而确保每种可能的起点都被考虑到。

\subsection*{结果输出}
\begin{itemize}
    \item 输出可以安全食用的最大水果数量。
    \item 输出所有选择的水果的ID。
    \item 输出这些水果的总价格。
\end{itemize}

\subsection*{特点和考虑}
\begin{itemize}
    \item 通过 \texttt{DFS} 探索所有可能的水果组合,能够有效处理组合优化问题。
    \item 使用剪枝技术,提高搜索效率,避免不必要的计算。
    \item 此算法在问题规模较大时可能会遇到性能瓶颈,但对于问题给定的规模(水果和提示数量都不超过100),这种方法是可行的。
\end{itemize}

\section*{代码}
\begin{lstlisting}[caption = main.cpp, language = c++]
#include <bits/stdc++.h>
using namespace std;
#define N 105
int numTips, numFruit;
bool deficiency[N][N]; // deficiency[i][j]=false 表示 i 和 j 不能一起吃。deficiency[i][j]=true 表示 i 和 j 可以一起吃。
int price[N];
int fruit[N]; // fruit[] 记录当前选择的水果的状态。sum 记录已选择的水果数量。
int ansFruit[N], ansNum = -1, ansMoney = 0; // ansFruit 记录最大水果集合,ansNum 记录最大数量的水果。ansMoney 记录最小金额。

int sumFruit[N]; // sumFruit[i]:如果我只从 'i 到 numFruit' 选择水果,我能选择的最大水果数。

void DFS(int pos, int sum, int money) // pos: 现在我选择了水果'pos',已经选择了 'sum' 数量的水果。money:当前花费的总金额。
{
    //if(now == numFruit) { CHECK(); return; }
    for(int i = pos + 1; i <= numFruit; ++i)
    {
        // 如果现在选择的水果数量加上从 i 到 numFruit 可以选择的最大水果数仍然小于 ansNum,我们可以忽略剩余的情况。
        if(sumFruit[i] + sum < ansNum) return;
        // 如果这个水果不能和已经选择的水果一起吃,则跳过。
        if(deficiency[i][pos] == false) continue;
        bool flag = true;
        for(int j = 0; j < sum; ++j) if(deficiency[fruit[j]][i] == false) { flag = false; break; }
        if(flag == false) continue;
        // 将这个水果添加到我们当前的选择中
        fruit[sum] = i;
        DFS(i, sum + 1, money + price[i]);
    }
    if(sum > ansNum) // 记录最大数量的水果和详细信息
    {
        ansNum = sum;
        for(int i = 0; i < sum; ++i) ansFruit[i] = fruit[i];
        ansMoney = money;
    }
    else if(sum == ansNum && money < ansMoney) // 检查这种选择是否可以用更少的钱
    {
        for(int i = 0; i < sum; ++i) ansFruit[i] = fruit[i];
        ansMoney = money;
    }
    return;
}

int main()
{
    memset(deficiency, 1, sizeof(deficiency));
    scanf("%d%d", &numTips, &numFruit);
    for(int i = 0, x, y; i < numTips; ++i)
    {
        scanf("%d%d", &x, &y);
        deficiency[x][y] = deficiency[y][x] = 0;
    }
    for(int i = 0, x, y; i < numFruit; ++i)
    {
        scanf("%d%d", &x, &y);
        price[x] = y;
    }
    
    for(int i = numFruit; i > 0; --i)
    {
        fruit[0] = i;
        DFS(i, 1, price[i]);
        sumFruit[i] = ansNum;
    }

    printf("%d\n", ansNum);
    printf("%03d", ansFruit[0]);
    for(int i = 1; i < ansNum; ++i) printf(" %03d", ansFruit[i]);
    printf("\n%d\n", ansMoney);
    return 0;
}

\end{lstlisting}

\subsection*{生成测试数据的程序}
\begin{lstlisting}[caption = 测试数据生成程序, language= python]
import random

def generate_test_data(num_tips, num_fruits):
    fruits = [f"{i:03d}" for i in range(1, num_fruits + 1)]
    tips = []
    prices = []

    # Generate tips (pairs of fruits that cannot be eaten together)
    while len(tips) < num_tips:
        f1, f2 = random.sample(fruits, 2)
        pair = f"{f1} {f2}"
        reverse_pair = f"{f2} {f1}"
        if pair not in tips and reverse_pair not in tips:
            tips.append(pair)
    
    # Generate prices for each fruit
    for fruit in fruits:
        price = random.randint(1, 100)  # Prices from 1 to 100
        prices.append(f"{fruit} {price}")

    return num_tips, num_fruits, tips, prices

# Example usage
num_tips = input("输入不能同时吃的水果数量: ")

num_fruits = input("输入所有的水果数量: ")
num_tips, num_fruits, tips, prices = generate_test_data(num_tips, num_fruits)

print(f"{num_tips} {num_fruits}")
for tip in tips:
    print(tip)
for price in prices:
    print(price)
\end{lstlisting}
    
\section*{测试}
\subsection*{测试1}
\begin{verbatim}
    16 20
    001 002
    003 004
    004 005
    005 006
    006 007
    007 008
    008 003
    009 010
    009 011
    009 012
    009 013
    010 014
    011 015
    012 016
    012 017
    013 018
    020 99
    019 99
    018 4
    017 2
    016 3
    015 6
    014 5
    013 1
    012 1
    011 1
    010 1
    009 10
    008 1
    007 2
    006 5
    005 3
    004 4
    003 6
    002 1
    001 2
\end{verbatim}

\subsection*{output1}
\begin{verbatim}
12
002 004 006 008 009 014 015 016 017 018 019 020
239
\end{verbatim}

\subsection*{input2}
\begin{verbatim}
    22 32
    014 017
    030 007
    022 008
    016 009
    014 016
    023 015
    005 017
    002 015
    019 009
    018 008
    012 011
    032 024
    016 020
    015 014
    029 013
    028 013
    004 013
    028 003
    003 023
    022 027
    005 007
    032 005
    001 93
    002 11
    003 72
    004 52
    005 5
    006 42
    007 27
    008 74
    009 80
    010 71
    011 43
    012 3
    013 65
    014 23
    015 60
    016 57
    017 34
    018 40
    019 71
    020 78
    021 35
    022 24
    023 78
    024 7
    025 98
    026 68
    027 76
    028 26
    029 61
    030 60
    031 48
    032 66
\end{verbatim}

\subsection*{output2}
\begin{verbatim}
21
001 002 004 005 006 010 012 014 018 019 020 021 022 023 024 025 026 028 029 030 031
994
\end{verbatim}

\section*{时间复杂度分析}
\subsection*{深度优先搜索 (DFS)}

在 \texttt{DFS} 函数中,对于每个可能的水果选择,都有一个递归调用,这个调用依赖于两个主要因素:当前位置 \texttt{pos} 和剩余可选的水果数量。在最坏的情况下,每次递归可以选择所有剩余的水果。设 $M$ 是水果的总数,最坏情况下,\texttt{DFS} 的时间复杂度是指数级的,大约为 $O(2^M)$。这是因为每个水果都有被选中或不被选中两种可能,形成了一个二叉决策树。

\subsection*{剪枝优化}

代码中使用了剪枝技术,尤其是在判断 $\texttt{sumFruit}[i] + \texttt{sum} < \texttt{ansNum}$ 时直接返回的逻辑。这种剪枝有助于减少不必要的递归调用,因为如果当前路径已经不可能超过已找到的最优解,就没有继续探索的必要。这种优化虽然有助于减少运行时间,但在最坏情况下仍然是指数级。



\end{document}