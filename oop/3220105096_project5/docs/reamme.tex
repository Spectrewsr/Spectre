\documentclass{ctexart} % 使用ctexart类,适合中文文档
\usepackage{xcolor} 
\usepackage{enumitem} % 加载enumitem包
\usepackage{verbatim}
\usepackage{listings}
\usepackage{xeCJK}
\usepackage{fontspec}



% 定义listings设置
\lstset{
    basicstyle=\ttfamily, % 使用等宽字体
    commentstyle=\color{gray}, % 注释颜色
    keywordstyle=\color{blue}, % 关键字颜色
    frame=single,
    stringstyle=\color{red}, % 字符串颜色
    breaklines=true, % 自动换行
    captionpos=b, % 标题位置在底部
    showstringspaces=false, % 不显示字符串中的空格标记
}

\title{8-5 Fraction}
\author{3220105096   \quad   王思睿}
\date{\today}

\begin{document}
\maketitle % 创建标题

添加了对分数进行自动约分的功能,并判断结果是否为0;
\vspace{10pt}

实现对输入格式不正确的报错。
\vspace{10pt}

\begin{lstlisting}[caption = 报错示例]
    输入不正确的格式,如输入2222报错
        Format error,should be 'numerator/denominator'

    输入的数字超过范围,如3333333333322222222/3报错
        Input value out of range: 3333333333322222222/3

    输入错误的分数,比如www/2, 报错示例
        Invalid input: www/2

    输入分母是0,报错
        Denominator shouldn't be 0!
\end{lstlisting}
\vspace{10pt}

这个project有如下基本要求:

\textbf{Write a class that represents a fraction number like 2/3.}

\textbf{The functions below have to be implemented for this class:}
\begin{enumerate}[itemsep = 0pt, parsep = 0pt]
    \item \textbf{Default ctor}
    \item \textbf{Ctor takes two integers as parameters}
    \item \textbf{Copy ctor}
    \item \textbf{Arithmetical operators: + - * /}
    \item \textbf{Relational operators: < <= == != >= >}
    \item \textbf{Typecast to double}
    \item \textbf{To string}
    \item \textbf{Inserter and extractor for streams}
    \item \textbf{Conversion from a finite decimal string like: 1.414}
\end{enumerate}

\vspace{10pt}
\begin{lstlisting}[language = c++, caption=1.Default ctor]
    // 构造函数:Fraction
    // 描述:默认构造函数,分子和分母都为1
    Fraction();
\end{lstlisting}

\vspace{10pt}
\begin{lstlisting}[language = c++, caption=2.Ctor takes two integers as parameters]
    Fraction::Fraction(int a, int b)
    {
        // 初始化分子和分母。
        this->denominator = b;
        this->numerator = a;
    }
\end{lstlisting}

\vspace{10pt}
\begin{lstlisting}[language = c++, caption=3.Copy ctor]
    // 拷贝构造函数,用于创建一个已存在分数的副本。
    Fraction::Fraction(const Fraction& other)
    {
        // 拷贝分子和分母。
        this->denominator = other.denominator;
        this->numerator = other.numerator;
    }
\end{lstlisting}

\vspace{10pt}
\begin{lstlisting}[language = c++, caption=4.Arithmetical operators: + - * /]
    // 运算符重载:+
    // 描述:实现分数的加法运算
    Fraction operator + (Fraction);

    // 运算符重载:-
    // 描述:实现分数的减法运算
    Fraction operator - (Fraction);

    // 运算符重载:*
    // 描述:实现分数的乘法运算
    Fraction operator * (Fraction);

    // 运算符重载:/
    // 描述:实现分数的除法运算
    Fraction operator / (Fraction);

\end{lstlisting}

\vspace{10pt}
\begin{lstlisting}[language = c++, caption={5.Relational operators: \textless, \textless=, ==, !=, \textgreater=, \textgreater}]
    // 运算符重载:< 
    // 描述:实现分数的小于比较运算
    bool operator < (const Fraction&);            

    // 运算符重载:<= 
    // 描述:实现分数的小于等于比较运算
    bool operator <= (const Fraction&);

    // 运算符重载:!=
    // 描述:实现分数的不等于比较运算
    bool operator != (const Fraction&);

    // 运算符重载:==
    // 描述:实现分数的等于比较运算
    bool operator == (const Fraction&);                    

    // 运算符重载:>
    // 描述:实现分数的大于比较运算
    bool operator > (const Fraction&);                

    // 运算符重载:>=
    // 描述:实现分数的大于等于比较运算
    bool operator >= (const Fraction&);              
\end{lstlisting}

\vspace{10pt}
\begin{lstlisting}[language = c++, caption=6.Typecast to double]
    // 将分数转换为双精度浮点数并输出。
    void fractionoper::frac_to_dou()
    {
        Fraction a;
        double a_double;
        std::string frac1;
        std::cout << "Enter your fraction:" << std::endl;

        std::cin >> frac1;
        try
        {
            a = Fraction(frac1); // 尝试将字符串转换为分数对象。
        }
        catch(const std::exception& e)
        {
            system("clear"); // 清屏。
            std::cerr << e.what() << '\n'; // 打印错误信息。
            return;
        }
        // 进行转换并输出结果。
        a_double = static_cast<double>(a.getnum()) / static_cast<double>(a.getden());
        std::cout << a_double << std::endl;
    }

\end{lstlisting}

\vspace{10pt}
\begin{lstlisting}[language = c++, caption=7.To string]
    // 将分数转换为字符串表示,如"1/2"。
    std::string Fraction::printstr()
    {
        std::stringstream ss;
        ss << this->numerator << "/" << this->denominator;
        return ss.str();
    }

\end{lstlisting}

\vspace{10pt}
\begin{lstlisting}[language = c++, caption=8.Inserter and extractor for streams]
    friend std::istream& operator>>(std::istream& in, Fraction& frac);
    friend std::ostream& operator<<(std::ostream& out, const Fraction& frac);     
    
    void fractionoper::oper_stream()
    {
        Fraction a;
        std::cout<<"Input a fraction\n";
        std::cin>>a;
        std::cout<<"The fraction is "<<a<<std::endl;
    }
\end{lstlisting}

\vspace{10pt}
\begin{lstlisting}[language = c++, caption=9.Conversion from a finite decimal string like: 1.414]
    void fractionoper::dou_to_frac()
    {
        std::string frac1;
        std::cout << "Enter your double number" << std::endl;
        getchar(); // 清理输入缓冲区。
        std::cin >> frac1;
        double num = std::stod(frac1); // 将字符串转换为double。
        int places = frac1.find('.'); // 查找小数点的位置。
        places = frac1.size() - places; // 计算小数位数。
        int numerator = static_cast<int>(num * pow(10, places)); // 转换为分子。
        int denominator = static_cast<int>(pow(10, places)); // 计算分母。
        int max_gcd = gcd(numerator, denominator); // 计算最大公约数以简化分数。
        // 输出转换结果。
        std::cout << "Conversion from a finite decimal string: " << frac1 << " to a fraction is " << numerator / max_gcd << "/" << denominator / max_gcd << std::endl;
    }
\end{lstlisting}

\vspace{10pt}

\large{ 效果展示:}
\begin{lstlisting}[caption = 测试add]
    Which operation would you like to take?
    1: Add
    2: Sub
    3: Mul
    4: Div
    5: Compare 2 fractions
    6: Turn a fraction to double
    7: Turn a double to fraction
    8: Turn a fraction to string
    9: Inserter and extractor for streams
    10: Exit

    1
    Enter your first fraction:
    3/3
    Enter your second fraction:
    -5/5
    The result of 3/3 + -5/5 is 0
    Which operation would you like to take?
    1: Add
    2: Sub
    3: Mul
    4: Div
    5: Compare 2 fractions
    6: Turn a fraction to double
    7: Turn a double to fraction
    8: Turn a fraction to string
    9: Inserter and extractor for streams
    10: Exit

    1
    Enter your first fraction:
    2/3
    Enter your second fraction:
    4/6
    The result of 2/3 + 4/6 is 4/3
    Which operation would you like to take?
    1: Add
    2: Sub
    3: Mul
    4: Div
    5: Compare 2 fractions
    6: Turn a fraction to double
    7: Turn a double to fraction
    8: Turn a fraction to string
    9: Inserter and extractor for streams
    10: Exit

    1
    Enter your first fraction:
    3/5
    Enter your second fraction:
    -7/10
    The result of 3/5 + -7/10 is -1/10
\end{lstlisting}
\vspace{10pt}


\begin{lstlisting}[caption = 测试sub]
    Which operation would you like to take?
    1: Add
    2: Sub
    3: Mul
    4: Div
    5: Compare 2 fractions
    6: Turn a fraction to double
    7: Turn a double to fraction
    8: Turn a fraction to string
    9: Inserter and extractor for streams
    10: Exit

    2
    Enter your first fraction:
    3/4
    Enter your second fraction:
    7/8 
    The result of 3/4 - 7/8 is -1/8
    Which operation would you like to take?
    1: Add
    2: Sub
    3: Mul
    4: Div
    5: Compare 2 fractions
    6: Turn a fraction to double
    7: Turn a double to fraction
    8: Turn a fraction to string
    9: Inserter and extractor for streams
    10: Exit

    2
    Enter your first fraction:
    5/10
    Enter your second fraction:
    1/2
    The result of 5/10 - 1/2 is 0
\end{lstlisting}
\vspace{10pt}

\begin{lstlisting}[caption = 测试mul]
    Which operation would you like to take?
    1: Add
    2: Sub
    3: Mul
    4: Div
    5: Compare 2 fractions
    6: Turn a fraction to double
    7: Turn a double to fraction
    8: Turn a fraction to string
    9: Inserter and extractor for streams
    10: Exit

    3
    Enter your first fraction:
    0/3
    Enter your second fraction:
    2/2
    The result of 0/3 * 2/2 is 0
    Which operation would you like to take?
    1: Add
    2: Sub
    3: Mul
    4: Div
    5: Compare 2 fractions
    6: Turn a fraction to double
    7: Turn a double to fraction
    8: Turn a fraction to string
    9: Inserter and extractor for streams
    10: Exit

    3
    Enter your first fraction:
    5/4
    Enter your second fraction:
    -6/8
    The result of 5/4 * -6/8 is -15/16
\end{lstlisting}
\vspace{10pt}

\begin{lstlisting}[caption = 测试div]
    Which operation would you like to take?
    1: Add
    2: Sub
    3: Mul
    4: Div
    5: Compare 2 fractions
    6: Turn a fraction to double
    7: Turn a double to fraction
    8: Turn a fraction to string
    9: Inserter and extractor for streams
    10: Exit

    4
    Enter your first fraction:
    4/5
    Enter your second fraction:
    -3/4
    The result of (4/5) / (-3/4) is 16/-15
    Which operation would you like to take?
    1: Add
    2: Sub
    3: Mul
    4: Div
    5: Compare 2 fractions
    6: Turn a fraction to double
    7: Turn a double to fraction
    8: Turn a fraction to string
    9: Inserter and extractor for streams
    10: Exit

    4
    Enter your first fraction:
    3/4
    Enter your second fraction:
    0/4
    The result of (3/4) / (0/4) is 0
\end{lstlisting}
\vspace{10pt}

\begin{lstlisting}[caption = 测试Compare 2 fractions]
    Which operation would you like to take?
    1: Add
    2: Sub
    3: Mul
    4: Div
    5: Compare 2 fractions
    6: Turn a fraction to double
    7: Turn a double to fraction
    8: Turn a fraction to string
    9: Inserter and extractor for streams
    10: Exit

    5
    Enter your first fraction:
    3/3
    Enter your second fraction:
    7/7
    3/3 == 7/7
    3/3 <= 7/7
    3/3 >= 7/7
    Which operation would you like to take?
    1: Add
    2: Sub
    3: Mul
    4: Div
    5: Compare 2 fractions
    6: Turn a fraction to double
    7: Turn a double to fraction
    8: Turn a fraction to string
    9: Inserter and extractor for streams
    10: Exit

    5
    Enter your first fraction:
    6/7
    Enter your second fraction:
    -3/4
    6/7 > -3/4
    6/7 >= -3/4
    6/7 != -3/4
\end{lstlisting}
\vspace{10pt}

\begin{lstlisting}[caption = 测试Turn a fraction to double]
    Which operation would you like to take?
    1: Add
    2: Sub
    3: Mul
    4: Div
    5: Compare 2 fractions
    6: Turn a fraction to double
    7: Turn a double to fraction
    8: Turn a fraction to string
    9: Inserter and extractor for streams
    10: Exit

    6
    Enter your fraction:
    4/6
    0.666667

    Which operation would you like to take?
    1: Add
    2: Sub
    3: Mul
    4: Div
    5: Compare 2 fractions
    6: Turn a fraction to double
    7: Turn a double to fraction
    8: Turn a fraction to string
    9: Inserter and extractor for streams
    10: Exit

    6
    Enter your fraction:
    -4/7
    -0.571429
\end{lstlisting}
\vspace{10pt}

\begin{lstlisting}[caption = 测试Turn a double to fraction]
    Which operation would you like to take?
    1: Add
    2: Sub
    3: Mul
    4: Div
    5: Compare 2 fractions
    6: Turn a fraction to double
    7: Turn a double to fraction
    8: Turn a fraction to string
    9: Inserter and extractor for streams
    10: Exit

    7
    Enter your double number
    1.414 
    Conversion from a finite decimal string: 1.414 to a fraction is 707/500

    Which operation would you like to take?
    1: Add
    2: Sub
    3: Mul
    4: Div
    5: Compare 2 fractions
    6: Turn a fraction to double
    7: Turn a double to fraction
    8: Turn a fraction to string
    9: Inserter and extractor for streams
    10: Exit

    7
    Enter your double number
    -1.414
    Conversion from a finite decimal string: -1.414 to a fraction is -707/500
\end{lstlisting}
\vspace{10pt}

\begin{lstlisting}[caption = 测试Turn a fraction to string]
    Which operation would you like to take?
    1: Add
    2: Sub
    3: Mul
    4: Div
    5: Compare 2 fractions
    6: Turn a fraction to double
    7: Turn a double to fraction
    8: Turn a fraction to string
    9: Inserter and extractor for streams
    10: Exit

    8
    Input your numerator and denominator:
    23 24
    The fraction to string is: 23/24
\end{lstlisting}
\vspace{10pt}

\begin{lstlisting}[caption = 测试Inserter and extractor for streams]
    Which operation would you like to take?
    1: Add
    2: Sub
    3: Mul
    4: Div
    5: Compare 2 fractions
    6: Turn a fraction to double
    7: Turn a double to fraction
    8: Turn a fraction to string
    9: Inserter and extractor for streams
    10: Exit

    9
    Input a fraction
    34/42
    The fraction is 34/42
\end{lstlisting}
\vspace{10pt}

\begin{lstlisting}[caption = 测试Exit]
    输入10表示exit
    程序会退出。
\end{lstlisting}
\vspace{10pt}
\end{document}