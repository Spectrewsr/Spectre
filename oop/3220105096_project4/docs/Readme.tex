\documentclass{ctexart} % 使用ctexart类,适合中文文档
\usepackage{xcolor} 
\usepackage{enumitem} % 加载enumitem包
\usepackage{verbatim}
\usepackage{listings}
\usepackage{xeCJK}
\usepackage{fontspec}



% 定义listings设置
\lstset{
    basicstyle=\ttfamily, % 使用等宽字体
    commentstyle=\color{gray}, % 注释颜色
    keywordstyle=\color{blue}, % 关键字颜色
    frame=single,
    stringstyle=\color{red}, % 字符串颜色
    breaklines=true, % 自动换行
    captionpos=b, % 标题位置在底部
    showstringspaces=false, % 不显示字符串中的空格标记
}



\title{project4}
\author{3220105096}
\date{\today}

\begin{document}

\maketitle % 创建标题
\textbf{\large \textcolor{red} {注意事项:} }
\begin{large}
\begin{itemize}[itemsep=0pt, parsep=0pt]
    \item \textcolor{red} {输入要求:一定要输入日期的方式是day34,day1,day5这种,
        day与数字之间不要有空格,日记的内容随便有空格,怎么写都行,比如day1 uuu iii mmm}
    \item 所有的可执行程序在bin中。
    \item cmake在build中。
    \item 总的测试脚本(diary\_manager.sh)在总目录下面。
    \item test文件夹下面的test1\_diary.sh中是一个测试脚本,直接执行可以在
            test文件夹下面的mydiary.txt中看到结果,并在屏幕上看到输出。
\end{itemize}
\end{large}


\vspace{10pt}

根据题目的要求,我实现的思路是首先实现一个Diary类,包含所有需要的函数,
然后根据题目的要求,有四个独立的程序,再用shell脚本对四个程序进行汇总,
通过shell脚本执行程序。实现了如果add有重复的日期插入,则会删掉以前那个日
期里面的东西换成心得,在原有位置不变,不会插入到末尾,但是所有没有过的日期默认插入到末尾。

\vspace{10pt}

首先我实现了dairy类,这里面包含add,remove,list,show方法,然后有4个独立的程序
分别调用了这些方法,实现了4个独立的程序。

\vspace{10pt}
这是我的Shell测试脚本代码:

\begin{lstlisting}[language=bash, caption=测试代码的脚本]
#!/bin/bash

# 定义可执行文件的路径
BIN_DIR="../bin"
DIARY_FILE="mydiary.txt"

# 清理现场,确保日记文件是空的
> "$DIARY_FILE"

# 添加条目
echo "Adding entries..."
echo "day1" "111 ffff" | "$BIN_DIR/pdadd"
echo "day3" "333 ttttt" | "$BIN_DIR/pdadd"
echo "day5" "555 eeeeee eeeee" | "$BIN_DIR/pdadd"

# 列出条目
echo "day1" "day3" | "$BIN_DIR/pdlist" "Listing entries..." 

# 展示条目
echo "day3" | "$BIN_DIR/pdshow" "Showing an entry..." 

# 移除条目
echo "day3" | "$BIN_DIR/pdremove" "Removing an entry..."

# 验证步骤

echo "Test completed."
\end{lstlisting}

这是执行后的txt文本
\begin{lstlisting}[caption={My Text}, captionpos=b]
    day1 111 ffff
    day5 555 eeeeeeeeeee
\end{lstlisting}
可以看出成功的将day3删掉了,并且没有任何多余的空行。

\vspace{10pt}
\vspace{10pt}

\begin{lstlisting}[language = bash, captionpos=b, caption = 最终脚本]
    #!/bin/bash

    # 定义可执行文件的目录路径
    BIN_DIR="./bin"

    # 执行相关程序的函数
    run_pdadd() {
        echo "Running pdadd"
        
        "$BIN_DIR/pdadd"
    }

    run_pdlist() {
        echo "Running pdlist"
        
        "$BIN_DIR/pdlist"
    }

    run_pdremove() {
        echo "Running pdremove"
        
        "$BIN_DIR/pdremove"
    }

    run_pdshow() {
        echo "Running pdshow"
        
        "$BIN_DIR/pdshow"
    }

    # 循环直到用户决定退出
    while true; do
        echo "Please choose an option:"
        echo "1. Add entry"
        echo "2. List entries"
        echo "3. Remove entry"
        echo "4. Show entry"
        echo "5. Exit"
        read -p "Enter your choice (1/2/3/4/5): " choice

        case $choice in
            1)
                run_pdadd
                ;;
            2)
                run_pdlist
                ;;
            3)
                run_pdremove
                ;;
            4)
                run_pdshow
                ;;
            5)
                echo "Exiting..."
                exit 0
                ;;
            *)
                echo "Invalid option, please try again."
                ;;
        esac
    done

\end{lstlisting}

下面是一段测试
\begin{lstlisting}[language = {}, caption={final test}, captionpos=b]
    Please choose an option:
    1. Add entry
    2. List entries
    3. Remove entry
    4. Show entry
    5. Exit
    Enter your choice (1/2/3/4/5): 1
    Running pdadd
    select a date
    day1
    the content you want to write:
    111 www
    Please choose an option:
    1. Add entry
    2. List entries
    3. Remove entry
    4. Show entry
    5. Exit
    Enter your choice (1/2/3/4/5): 1
    Running pdadd
    select a date
    day2
    the content you want to write:
    yyy yyyyyyyyyyyyyyyyyyyyyyyyyyyyy
    Please choose an option:
    1. Add entry
    2. List entries
    3. Remove entry
    4. Show entry
    5. Exit
    Enter your choice (1/2/3/4/5): 1
    Running pdadd
    select a date
    day3
    the content you want to write:
    uuuuuuuuuuuuuu uuuuuuuuuu
    Please choose an option:
    1. Add entry
    2. List entries
    3. Remove entry
    4. Show entry
    5. Exit
    Enter your choice (1/2/3/4/5): 1  
    Running pdadd
    select a date
    day7
    the content you want to write:
    yyyyyy
    Please choose an option:
    1. Add entry
    2. List entries
    3. Remove entry
    4. Show entry
    5. Exit
    Enter your choice (1/2/3/4/5): 1
    Running pdadd
    select a date
    day9
    the content you want to write:
    99999999
    Please choose an option:
    1. Add entry
    2. List entries
    3. Remove entry
    4. Show entry
    5. Exit
    Enter your choice (1/2/3/4/5): 1    
    Running pdadd
    select a date
    day11
    the content you want to write:
    jjj 666
    Please choose an option:
    1. Add entry
    2. List entries
    3. Remove entry
    4. Show entry
    5. Exit
    Enter your choice (1/2/3/4/5): 2
    Running pdlist
    select the begin date and the end date
    day2
    day9
    day2 yyy yyyyyyyyyyyyyyyyyyyyyyyyyyyyy
    day3 uuuuuuuuuuuuuu uuuuuuuuuu
    day7 yyyyyy
    day9 99999999
    Please choose an option:
    1. Add entry
    2. List entries
    3. Remove entry
    4. Show entry
    5. Exit
    Enter your choice (1/2/3/4/5): 3
    Running pdremove
    the date you want to remove
    day3
    Please choose an option:
    1. Add entry
    2. List entries
    3. Remove entry
    4. Show entry
    5. Exit
    Enter your choice (1/2/3/4/5): 2
    Running pdlist
    select the begin date and the end date
    day2 day9
    day2 yyy yyyyyyyyyyyyyyyyyyyyyyyyyyyyy
    day7 yyyyyy
    day9 99999999
    Please choose an option:
    1. Add entry
    2. List entries
    3. Remove entry
    4. Show entry
    5. Exit
    Enter your choice (1/2/3/4/5): 1
    Running pdadd
    select a date
    day2
    the content you want to write:
    daydayday
    Please choose an option:
    1. Add entry
    2. List entries
    3. Remove entry
    4. Show entry
    5. Exit
    Enter your choice (1/2/3/4/5): 4
    Running pdshow
    the date you want to show
    day2
    day2 daydayday
    Please choose an option:
    1. Add entry
    2. List entries
    3. Remove entry
    4. Show entry
    5. Exit
    Enter your choice (1/2/3/4/5):
\end{lstlisting}
% 这里可以添加你的文档内容

\end{document}
