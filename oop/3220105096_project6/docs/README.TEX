\documentclass{ctexart} % 使用ctexart类,适合中文文档
\usepackage{xcolor} 
\usepackage{enumitem} % 加载enumitem包
\usepackage{verbatim}
\usepackage{listings}
\usepackage{xeCJK}
\usepackage{fontspec}
\usepackage{graphicx}
\usepackage{hyperref}
\usepackage{cleveref}
\usepackage{nameref}


% 定义listings设置
\lstset{
    basicstyle=\ttfamily, % 使用等宽字体
    commentstyle=\color{gray}, % 注释颜色
    keywordstyle=\color{blue}, % 关键字颜色
    frame=tb,
    stringstyle=\color{red}, % 字符串颜色
    numbers = left,
    breaklines=true, % 自动换行
    captionpos=b, % 标题位置在底部
    showstringspaces=false, % 不显示字符串中的空格标记
    numberstyle=\tiny\color{green},
}

\title{project6}
\author{3220105096}
\date{\today}

\begin{document}
\maketitle % 创建标题

\section{题目要求}

Given the declaration of a class template Vector as shown below, implement the bodies of all the member functions. Please also write a main function to test all the facilities of the class Vector.
\begin{lstlisting}[language = c++, caption = Vector]
template <class T>
class Vector {
public:
  Vector();                      // creates an empty vector
  Vector(int size);              // creates a vector for holding 'size' elements
  Vector(const Vector& r);       // the copy ctor
  ~Vector();                     // destructs the vector 
  T& operator[](int index);      // accesses the specified element without bounds checking
  T& at(int index);              // accesses the specified element, throws an exception of
                                 // type 'std::out_of_range' when index <0 or >=m_nSize
  int size() const;              // return the size of the container
  void push_back(const T& x);    // adds an element to the end 
  void clear();                  // clears the contents
  bool empty() const;            // checks whether the container is empty 
private:
  void inflate();                // expand the storage of the container to a new capacity,
                                 // e.g. 2*m_nCapacity
  T *m_pElements;                // pointer to the dynamically allocated storage
  int m_nSize;                   // the number of elements in the container
  int m_nCapacity;               // the total number of elements that can be held in the
                                 // allocated storage
};
\end{lstlisting}

\section{函数实现}
\subsection{构造函数的实现}

\begin{lstlisting}[language = c++, caption = Vector()]
    Vector<T>::Vector()
    {
        m_nSize = 0;                   
        m_nCapacity = 1;             
        m_pElements = new T[m_nCapacity]; 
    }
    
    template <class T>
    Vector<T>::Vector(int size)
    {
        this->m_nSize = size;
        this->m_nCapacity = size;
        this->m_pElements = new T[size];
    }
    
    template <class T>
    Vector<T>::Vector(const Vector& r)
    {
        this->m_nSize = r.m_nSize;
        this->m_nCapacity = r.m_nCapacity;
        this->m_pElements = new T[m_nCapacity];
        for ( int i = 0; i < m_nSize; ++i )
        {
            m_pElements[i] = r.m_pElements[i];
        }
    }
\end{lstlisting}

\subsection{析构函数}
\begin{lstlisting}[caption = ~Vector, language = c++]
    template <class T>
    Vector<T>::~Vector()
    {
    
        delete[] m_pElements;
        m_nCapacity = 0;
        m_nSize = 0;
        m_pElements = nullptr;
    }
\end{lstlisting}

\subsection{私有函数inflate()}
\begin{lstlisting}[caption = inflate(), language = c++]
    template <class T>
    void Vector<T>::inflate()
    {
        if(m_nCapacity == 0)
        {
            m_nCapacity = 1;
        }
        else
        {
            m_nCapacity *= 2 ;
        }
    
        T* newElements = new T[m_nCapacity];
        for (int i = 0; i < m_nSize; i++) 
        {
            newElements[i] = m_pElements[i];
        }
    
        delete[] m_pElements;
        m_pElements = newElements;
    }
    
\end{lstlisting}

\subsection{公有函数}
\begin{lstlisting}[caption = public, language = c++]
    template<class T>
    int Vector<T>::size() const
    {
        return m_nSize;
    }

    template <class T>
    T& Vector<T>::operator[](int index) 
    {
        return m_pElements[index];
    }

    template <class T>
    T& Vector<T>::at(int index) 
    {
        if (index < 0 || index >= m_nSize) 
        {
            throw std::out_of_range("Index out of range");
        }
        return m_pElements[index];
    }
    template <class T>
    bool Vector<T>::empty() const
    {
        return m_nSize == 0;
    }

    template <class T>
    void Vector<T>::clear()
    {
        m_nCapacity = 0;
        m_nSize = 0;
        m_pElements = nullptr;
        delete[] m_pElements;
    }

    template <class T>
    void Vector<T>::push_back(const T& x)
    {
        if (m_nSize == m_nCapacity) 
        {
            inflate();
        }
        m_pElements[m_nSize++] = x;
    }

#endif
\end{lstlisting}

\section{测试}
\begin{lstlisting}[language = c++, caption = main]
    Vector <int> a1;
    int num;
    for ( int i = 0; i < 100; ++i )
    {
       
        a1.push_back(i);
    }

    for ( int i = 0; i < 100; ++i )
    {
        std::cout<<a1[i]<<" ";
    }
    std::cout<<std::endl;
    //std::cout<<a1.at(1000)<<std::endl;


    Vector <int> a2(a1);
    std::cout<<a2.size()<<std::endl;

    Vector <int> a3(10000);
    std::cout<<a3.size()<<std::endl;


    Vector <std::string> str;
    std::string s0;
    std::cout<<"Input the first string"<<std::endl;
    std::cin>>s0;
    str.push_back(s0);
    std::cout<<str[0]<<std::endl;
    std::cout<<"Input the Second string"<<std::endl;
    std::cin>>s0;
    str.push_back(s0);
    std::cout<<str[1]<<std::endl;

    a1.~Vector();
    a2.clear();
    str.clear();

    std::cout<<a1.size()<<std::endl;
    std::cout<<a2.size()<<std::endl;
    std::cout<<str.size()<<std::endl;
\end{lstlisting}

首先Vector a1:
a1的预先设置内存为1
\begin{lstlisting}
    Vector<T>::Vector()
    {
        m_nSize = 0;                   
        m_nCapacity = 1;             
        m_pElements = new T[m_nCapacity]; 
    }
\end{lstlisting}
对a1进行100次插入操作。超出原有的内存,自动建立新的内存,说明inflate执行成功。

对a1中的内容执行输出操作,0 1 2 3 4 5 6 7 8 9 10 11 12 13 14 15 16 17 18 19 20 21 22 23 24 25 26 27 28 29 30 31 32 33 34 35 36 37 38 39 40 41 42 43 44 45 46 47 48 49 50 51 52 53 54 55 56 57 58 59 60 61 62 63 64 65 66 67 68 69 70 71 72 73 74 75 76 77 78 79 80 81 82 83 84 85 86 87 88 89 90 91 92 93 94 95 96 97 98 99


\vspace{10pt}

说明push\_back函数正常运行。\

测试第二个构造函数,
\vspace*{1pt}

Vector <int> a3(10000);
\vspace*{1pt}

std::cout<<a3.size()<<std::endl;
\vspace*{1pt}
输出为10000,说明第二个构造函数正常运行。

测试第三个构造函数:
\vspace*{1pt}

Vector <int> a2(a1);
\vspace*{1pt}

std::cout<<a2.size()<<std::endl;
\vspace*{1pt}

for ( int i = 0; i < 100; ++i )
{
    std::cout<<a2[i]<<" ";
}
\vspace*{1pt}

输出0 1 2 3 4 5 6 7 8 9 10 11 12 13 14 15 16 17 18 19 20 21 22 23 24 25 26 27 28 29 30 31 32 33 34 35 36 37 38 39 40 41 42 43 44 45 46 47 48 49 50 51 52 53 54 55 56 57 58 59 60 61 62 63 64 65 66 67 68 69 70 71 72 73 74 75 76 77 78 79 80 81 82 83 84 85 86 87 88 89 90 91 92 93 94 95 96 97 98 99

说明第三个构造函数正确

测试at
\vspace*{1pt}

std::cout<<a1.at(1000)<<std::endl;
\vspace*{1pt}

抛出报错:
% 文本之前添加空间
\vspace{10pt}

\noindent 输出错误消息:
\begin{verbatim}
terminate called after throwing an instance of std::out_of_range
what():  Index out of range
\end{verbatim}

% 文本之后添加空间
\vspace{10pt}

说明正确。

测试其他类和operator[]:
\begin{lstlisting}
    Vector <std::string> str;
    std::string s0;
    std::cout<<"Input the first string"<<std::endl;
    std::cin>>s0;
    str.push_back(s0);
    std::cout<<str.operator[](0)<<std::endl;
    std::cout<<"Input the Second string"<<std::endl;
    std::cin>>s0;
    str.push_back(s0);
    std::cout<<str.operator[](1)<<std::endl;
\end{lstlisting}
输出为:
\begin{verbatim}
Input the first string
zju
zju
Input the Second string
oop
oop
\end{verbatim}

说明Vector在类之间通用,并且可以使用operator[]


\vspace{10pt}

最后测试析构函数和clear函数
\begin{verbatim}
    a1.~Vector();
    a2.clear();
    str.clear();

    std::cout<<a1.size()<<std::endl;
    std::cout<<a2.size()<<std::endl;
    std::cout<<str.size()<<std::endl;
\end{verbatim}

输出为
\begin{verbatim}
0
0
0
\end{verbatim}
说明clear函数与析构函数正确。
\end{document}